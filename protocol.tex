
\documentclass{scrreprt}

\usepackage[utf8]{inputenc}
\usepackage{zed-csp}

\begin{document}

\author{Andreas Krohn, Benjamin Jochheim, Theodor Nolte, Benjamin Vetter}
\title{TH2 - Übung 2}

\maketitle

\chapter{CSP Basis}

\section{Alphabete}

\begin{equation}
  P = (a \then b \then Skip) \extchoice (b \then d \then Stop)
\end{equation}

\begin{eqnarray*}
  \alpha(P) &=& \alpha(a \then b \then Skip) \cup \alpha(b \then d \then Stop) \\
            &=& \alpha(b \then Skip) \cup \{a\} \cup \alpha(d \then Stop) \cup \{b\} \\
            &=& \alpha(Skip) \cup \{b\} \cup \{a\} \cup \alpha(Stop) \cup \{d\} \cup \{b\} \\
            &=& \emptyset \cup \{b, a\} \cup \emptyset \cup \{b, d\} \\
            &=& \{a, b, d\}
\end{eqnarray*}

\begin{equation}
  Q = (x \then y \then Stop) \intchoice (u \then Stop)
\end{equation}

\begin{eqnarray*}
  \alpha(Q) &=& \alpha(x \then y \then Stop) \cup \alpha(u \then Stop) \\
            &=& \alpha(y \then Stop) \cup \{x\} \cup \alpha(Stop) \cup \{u\} \\
            &=& \alpha(Stop) \cup \{y\} \cup \{x\} \cup \emptyset \cup \{u\} \\
            &=& \emptyset \cup \{x, y, u\} \\
            &=& \{u, x, y\}
\end{eqnarray*}

\begin{equation}
  R = (Q; P) \setminus \{x, y\}
\end{equation}

\begin{eqnarray*}
  \alpha(R) &=& \alpha(Q; P) \setminus \{x, y\} \\
            &=& (\alpha(Q) \cup \alpha(P)) \setminus \{x, y\} \\
            &=& (\{u, x, y\} \cup \{a, b, d\}) \setminus \{x, y\} \\
            &=& \{a, b, d, u, x, y\} \setminus \{x, y\} \\
            &=& \{a, b, d, u\}
\end{eqnarray*}

\begin{equation}
  S = (P \parallel[\{a, b\}][\{x, y\}] R) \bigtriangleup Q
\end{equation}

\begin{eqnarray*}
  \alpha(S) &=& \alpha(P \parallel[\{a, b\}][\{x, y\}] R) \cup \alpha(Q) \\
            &=& \{a, b\} \cup \{x, y\} \cup \{u, x, y\} \\
            &=& \{a, b, u, x, y\}
\end{eqnarray*}


\chapter{CSP Modelle}
\section{System Zusammensetzen}

Betrachten Sie das Fragment einer CSP-Spezifikation für eine Variante des Peterson-Algorithmus für beliebig viele Prozesse in der Datei \texttt{peterson-skeleton.csp}. Ergänzen Sie die Spezifikation durch
\begin{itemize}
 \item a) durch passende Kanaldeklarationen für die verwendeten Kanäle
 \item b) durch eine Deklaration für das System \texttt{SYS} für 3 Prozesse mit der Variablen \texttt{turn} und dem Mengen-Prozess \texttt{interested}
 \item c) durch geeignete Prüfanweisungen, die prüfen, ob das System den wechselseitigen Ausschluss garantieren.
 \item d) durch geeignete Prüfanweisungen, die belegen, dass Petersons Lösung keine strict alternation realisiert.
\end{itemize}
Demonstrieren Sie die Prüfungen und die Spezifikation im Praktikum.

\end{document}

