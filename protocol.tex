
\documentclass{scrreprt}

\usepackage{amsmath, amsthm, amssymb}
\usepackage[utf8]{inputenc}
\usepackage{zed-csp}

\begin{document}

\author{Andreas Krohn, Benjamin Jochheim, Theodor Nolte, Benjamin Vetter}
\title{TH2 - Übung 2}

\maketitle

\chapter{CSP Basis}

\section{Alphabete}

\begin{equation}
  P = (a \then b \then Skip) \extchoice (b \then d \then Stop)
\end{equation}

\begin{eqnarray*}
  \alpha(P) &=& \alpha(a \then b \then Skip) \cup \alpha(b \then d \then Stop) \\
            &=& \alpha(b \then Skip) \cup \{a\} \cup \alpha(d \then Stop) \cup \{b\} \\
            &=& \alpha(Skip) \cup \{b\} \cup \{a\} \cup \alpha(Stop) \cup \{d\} \cup \{b\} \\
            &=& \emptyset \cup \{b, a\} \cup \emptyset \cup \{b, d\} \\
            &=& \{a, b, d\}
\end{eqnarray*}

\begin{equation}
  Q = (x \then y \then Stop) \intchoice (u \then Stop)
\end{equation}

\begin{eqnarray*}
  \alpha(Q) &=& \alpha(x \then y \then Stop) \cup \alpha(u \then Stop) \\
            &=& \alpha(y \then Stop) \cup \{x\} \cup \alpha(Stop) \cup \{u\} \\
            &=& \alpha(Stop) \cup \{y\} \cup \{x\} \cup \emptyset \cup \{u\} \\
            &=& \emptyset \cup \{x, y, u\} \\
            &=& \{u, x, y\}
\end{eqnarray*}

\begin{equation}
  R = (Q; P) \setminus \{x, y\}
\end{equation}

\begin{eqnarray*}
  \alpha(R) &=& \alpha(Q; P) \setminus \{x, y\} \\
            &=& (\alpha(Q) \cup \alpha(P)) \setminus \{x, y\} \\
            &=& (\{u, x, y\} \cup \{a, b, d\}) \setminus \{x, y\} \\
            &=& \{a, b, d, u, x, y\} \setminus \{x, y\} \\
            &=& \{a, b, d, u\}
\end{eqnarray*}

\begin{equation}
  S = (P \parallel[\{a, b\}][\{x, y\}] R) \bigtriangleup Q
\end{equation}

\begin{eqnarray*}
  \alpha(S) &=& \alpha(P \parallel[\{a, b\}][\{x, y\}] R) \cup \alpha(Q) \\
            &=& \{a, b\} \cup \{x, y\} \cup \{u, x, y\} \\
            &=& \{a, b, u, x, y\}
\end{eqnarray*}


\chapter{CSP Modelle}
\section{System Zusammensetzen}

Betrachten Sie das Fragment einer CSP-Spezifikation für eine Variante des Peterson-Algorithmus für beliebig viele Prozesse in der Datei \texttt{peterson-skeleton.csp}. Ergänzen Sie die Spezifikation durch
\begin{itemize}
 \item a) durch passende Kanaldeklarationen für die verwendeten Kanäle
 \item b) durch eine Deklaration für das System \texttt{SYS} für 3 Prozesse mit der Variablen \texttt{turn} und dem Mengen-Prozess \texttt{interested}
 \item c) durch geeignete Prüfanweisungen, die prüfen, ob das System den wechselseitigen Ausschluss garantieren.
 \item d) durch geeignete Prüfanweisungen, die belegen, dass Petersons Lösung keine strict alternation realisiert.
\end{itemize}
Demonstrieren Sie die Prüfungen und die Spezifikation im Praktikum.

\section{Debugging bei CSP}

In der Datei rauchmelder-problems.csp finden Sie ein CSP Prozess-System, das
einen Rauchmelder mit zwei Sensorkomponenten und einer zentralen Meldeeinheit
versucht entsprechend der folgenden informellen Spezikfaiton zu modellieren:

\begin{itemize}
  \item Die Sensorkomponenten können per {\tt reset}-Kommando zurückgesetzt
        werden.
  \item Die Meldeeinheit fragt bei den Sensorkomponenten regelmäßig mit dem
        Kommando {\tt check} an, ob Rauch detektiert wird.
  \item Auf diese Anfrage der Zentraleinheit liefert eine Sensorkomponente
        zurück, ob sie Rauch detektiert ({\tt smoke}) oder nicht ({\tt no smoke}).
  \item Nachdem eine Sensorkomponente smoke gemeldet hat, tut sie dies auf alle
        folgenden Anfragen bis zum Zurücksetzen.
  \item Die zentrale Meldeeinheit geht erst bei einer Positivmeldung von beiden
        Sensoren davon aus, dass eine Gefahrensituation vorliegt. Nach Empfang eines
        {\tt smoke} Signals fragt sie daher bei dem zweiten Sensor an, ob dieser
        auch Rauch detektiert.
  \item In dem Fall, in dem beide Sensoreinheiten smoke melden, wird von der
        Meldeeinheit ein {\tt alarm} Signal ausgelöst.
  \item Nachdem die Meldeeinheit auf das alarm Signal ein {\tt ack} Signal aus der
        Umgebung erhalten hat, setzt sie die Sensoreinheiten zurück.
  \item In dem Fall, dass nur eine Sensoreinheit {\tt smoke} meldet, wird dies als
        Fehler angesehen und die Komponente zurückgesetzt.
\end{itemize}

Die Spezifikation wie sie vorliegt ist fehlerhaft und zwar sowohl bzgl der
Syntax, der statischen Semantik, der Modellumsetzung und der Prüfeigenschaften.
Finden Sie die Fehler mit FDR und dokumentieren und korrigieren Sie sie. Bei
der Dokumentation der Fehler geben Sie bitte an

\begin{itemize}
  \item wie sie den Fehler entdeckt haben
  \item wie sich der Fehler bemerkbar macht
  \item wie er behoben wurde
\end{itemize}

Lösung s. Ausdruck

\section{Verifikation mit Refinement}

In Restaurantküchen gibt es die Vorschrift, dass zwischen der Verarbeitung von
rohem und gebratenem Fleisch die Hände gewaschen werden müssen. Es wurde
versucht, diese Regel in der folgenden CSP-Spezifikation zu erfassen:

\begin{eqnarray*}
  ROH &=& roh \then waschen \then ROH \\
  BRATEN &=& waschen \then braten \then BRATEN \\
  SYS &=& ROH \parallel[\set{waschen}] BRATEN
\end{eqnarray*}

Prüfen Sie ob dieser Prozess die beschriebene Regel wiedergibt mit Hilfe von CSP-Refinement,
indem Sie mit einen geeigneten Prozess X eine Refinementcheck im Failuresmodell durchführt.
Wie muss X aussehen und wie die Verfeinerungsrelation?
Demonstrieren Sie im Praktikum die Spezifikation und die Prüfung.

\begin{equation*}
  X = (roh \then waschen \then X) \intchoice (braten \then waschen \then X)
\end{equation*}

\begin{equation*}
  X \refinedby_F SYS
\end{equation*}

$x \refinedby_F SYS$ ist falsch. $SYS$ erlaubt bspw. $roh \then waschen \then braten \then roh$,
was offensichtlich ein Verstoß gegen die Vorschrift ist, dass zwischen der Verarbeitung von
rohem und gebratenem Fleisch die Hände gewaschen werden müssen.

\end{document}

